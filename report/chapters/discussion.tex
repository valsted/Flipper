%What are the strengths and shortcomings of your device? Did it match the requirements?  How would you improve/develop it further, if you had time? If you had to produce your device in a factory for mass production, what would you modify? 

One of the biggest strength of our pinball machine is the almost complete independence of the different components. 
Since each component is independent, it is incredible easy to change and add new components to the board. This was also part of the motivation for choosing this project, that it was easy to add new components and effects to the project .\\

If we continued working on the project we do have some things we would like to change; 

First we tried to add a sensor by the hole, where the ball is returned to launch place, to create a way for the player to lose the game. However, when we tested the sensor it became obvious that we could not get any useful readings for it, in the environment we had chosen for it. Unfortunately it was so late in the project that we did not have time to try and fix this problem. This would probably be a pretty simple fix that would improve the player experience.

It would also be preferable to add more ways for the player to get point. An easy one would be to make the ramp be part of the point system by adding an sensor or a switch.

Finally it would be nice to add a bigger display so it would be easier for the player to keep track of their lives, score and other important player information e.g. high scores.\\

If we needed to mass produce this project it would first and furthermost be a great idea simulate the game. This way we could make sure that all components could be activated, which placements are best and that the game has a good flow.

We have multiple incidents where this would have benefited us, especially regarding placement of the components. A good example of this is the ramp, which had to be moved in order for us to hit it at the right angle.

We also relied heavenly on 3D-printing for a lot of our components, which would needed to be switched out with other more efficient methods such as molding or laser-cutting. The holes in the board for screws was also added along the way manually, this should be done when the playing field is laser-cut. 

%Another thing would be to create PCBs for all the circuits now that we know the components are working and the circuits are final.